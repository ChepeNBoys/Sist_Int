\documentclass[12pt, letterpaper]{article}
\usepackage[utf8]{inputenc}
\usepackage[spanish]{babel}
\usepackage{multicol}

\title{Metodos para traducción automatica}
\author{Mateo Gonzalez Ocampo \and Juan Alejandro Uribe Ramirez}
\date{Junio 2020}

\begin{document}
    \maketitle
    \begin{abstract}
        En el siguiente trabajo se realiza una revisión de algunos de los metodos que se han usado historicamente para resolver el problema de la 
        traducción automatica. Los metodos seleccionados fueron la traducción basada en reglas, basada en ejemplos y basada en redes neuronales.
        Para cada metodo se presenta una pequeña introducción historica, un analisis de como funciona y algunos resultados obtenidos con estos
         metodos, así como un analisis de sus ventajas y desventajas. Finalmente se presentan algunas conclusiones acerca de lo encontrado en
         la literatura durante la investigación de estos metodos. 
    \end{abstract}
    \begin{multicols}{2}
        \section*{Introducción}
            El ser humano ha buscado desarrollar herramientas que le permitan codificar el lenguaje para su transmisión, procesamiento y entendimiento 
            con el fin de llevar a cabo tareas útiles. La invención del teléfono marcó el punto de inicio en el reconocimiento moderno del lenguaje, 
            en tanto permitió cambiar el vehículo natural del lenguaje de mecánico a eléctrico. Homer Dudley’s llevó a cabo los primeros intentos para 
            sintetizar eléctricamente discursos en la década 1920-1930.

            No obstante, los avances necesarios para hacer que lo anterior fuera posible se dieron en la década de 1940-1950 cuando se sentaron las 
            primeras nociones de la teoría de información. Posteriormente, se dieron avances el hardware necesario para medir y digitalizar señales 
            sonoras y finalmente, la descripción de señales sonoras en términos de coeficientes de predicción linear (LPC), lo que proporcionó una 
            representación conveniente para este objetivo [1].

            Desde una perspectiva de alto nivel, el Procesamiento del Lenguaje Natural desarrolla aplicaciones que facilitan las interacciones entre 
            computadores y el lenguaje humano, partiendo del modelado de mecanismos cognitivos que lo entiendan y lo produzcan.

            Sus aplicaciones incluyen, entre otras, el reconocimiento de expresiones, análisis léxico, análisis gramatical, traducción, respuesta a 
            preguntas, análisis de sentimientos y generación de lenguaje natural.

            Desde una perspectiva histórica, el Procesamiento del Lenguaje Natural se ha discutido desde diferentes metodologías, las cuales se pueden 
            resumir en tres olas [2]:
                • Racionalismo.
                • Empirismo.
                • Aprendizaje profundo.
            La Figura 1 muestra los hitos que han constituido un avance en el Procesamiento del Lenguaje Natural a lo largo de su historia.        
    \end{multicols}
    
\end{document}